\documentclass[aspectratio=169]{beamer}

\usepackage{graphicx}
\usepackage{outlines}
\usepackage{multicol}

\usetheme{PaloAlto}
\usecolortheme{whale}

\addtobeamertemplate{footline}
{%
  \usebeamercolor[fg]{title in sidebar}
  \vskip-1cm\hskip10pt
  \insertframenumber\,/\,\inserttotalframenumber\kern1em\vskip2pt%
}

\begin{document}

%----------BEGIN TITLE----------

\title{Nuclear War City Simulator 2019 Xtreme Edition XXL}
\author[]{Josef Bostik \\
  Thomas van Haastrecht \\
  Eric Pereira \\
  Ryan Wojtyla}
\date{\today}

\begin{frame}
  \titlepage
\end{frame}

%-----------END TITLE-----------

\section{Origins}

%----------BEGIN ORIGINS----------

\begin{frame}

  \frametitle{Initial Ruminations}

  \begin{center}
    \includegraphics[scale=0.5]{../Diagrams/Game/Game.png}
  \end{center}

\end{frame}

%-----------END ORIGINS-----------

%----------BEGIN KIND OF GAME----------

\begin{frame}

  \frametitle{Settlement}

  \begin{multicols}{2}

  \begin{outline}
    \1 turn-based strategy
    \1 2D grid
    \1 rogue-like
      \2 short
      \2 procedurally generated (we hope)
      \2 high replayability
  \end{outline}

  \columnbreak

  \begin{center}
    \includegraphics[scale=0.3]{../Images/civLogo.png}
  \end{center}

  \begin{center}
    \includegraphics[scale=0.25]{../Images/FTLLogo.png}
  \end{center}

  \end{multicols}

\end{frame}

%-----------END KIND OF GAME-----------

\section{Game Engine}

%----------BEGIN GAME ENGINE----------

\begin{frame}

  \frametitle{Game Engine}

  \begin{center}
    \includegraphics[scale=0.7]{../Diagrams/GameEngine/GameEngine.png}
  \end{center}

\end{frame}

%-----------END GAME ENGINE-----------

%----------BEGIN USING GODOT----------

\begin{frame}

  \frametitle{Working with Godot}

  \begin{outline}
    \1 Steep initial learning curve
    \1 Official Godot documentation is expansive, but it lacks examples.
    \1 Community tutorials are more example-based.
    \1 Recent updates have obsolesced a significant portion of provided
    community support.
      \2 solutions for the old system
  \end{outline}

  \vspace*{1cm}

  \begin{outline}
    \1 The learning curve soon plateaus.
    \1 The native scripting language, GDScript, is nice to use.
    \1 An extensive collection of objects is provided.
  \end{outline}

\end{frame}

%-----------END USING GODOT-----------

\section{Foundations}

%----------BEGIN PREMISE----------

\begin{frame}

  \frametitle{Story}

  \begin{center}
    {\Large Prepare for imminent global thermonuclear war!}
  \end{center}

  \begin{outline}
    \1 Ensure the survival of as many of your citizens as possible.
    \1 Pacify the masses.
    \1 Manage resources.
    \1 Modify infrastructure.
  \end{outline}

  \begin{center}
    \includegraphics[scale=0.15, trim=6cm 0 0 6cm]{../Images/nuclear.png}
    \includegraphics[width=0.35\textwidth, trim=0 0 4cm 6cm]{../Images/board.png}
  \end{center}

\end{frame}

%-----------END PREMISE-----------

\section{Pieces}

\subsection{Population}

%----------BEGIN POPULATION----------

\begin{frame}

  \frametitle{Population}

  \begin{center}
    {\large Who really wants to live anyway?}
  \end{center}

  \begin{outline}
    \1 Choose whether or not to keep your citizens informed.
    \1 Relocate your citizens.
    \1 Don't push them too hard!
    \1 Keep them happy.
  \end{outline}

  \begin{center}
    \includegraphics[scale=2.0]{../reset/images/happiness.png}
  \end{center}

\end{frame}

%-----------END POPULATION-----------

\subsection{Resources}

%----------BEGIN RESOURCES----------

\begin{frame}

  \frametitle{Resources}

  \begin{center}
    {\large Everything needed to keep your citizens alive.}
  \end{center}

  \begin{outline}
    \1 Optimize the distribution and allocation of limited critical resources.
    \1 Maintain citizen complacency to ensure swift movement of supplies.  
    \1 Plan for the future.
  \end{outline}

  \begin{center}
    \includegraphics[scale=3.0]{../Images/factory.png}
  \end{center}

\end{frame}

%-----------END RESOURCES-----------

\subsection{Buildings}

%----------BEGIN BUILDINGS OVERVIEW----------

\begin{frame}

  \frametitle{Buildings - Overview}

  \begin{center}
    {\large Safe haven or gravestone?}
  \end{center}

  %\begin{multicols}{2}

  \begin{outline}
    \1 Buildings play several unique and critical roles within the city.
    \1 Store resources.
    \1 Maintain buildings to help keep the peace.
    \1 Fortify buildings to increase their chance of survival.
    \1 Production buildings can aid the surviving population.
  \end{outline}

  %\columnbreak

  %\begin{center}
  %  \includegraphics[scale=0.34, trim=4cm 0 4.5cm
  %  0]{../Diagrams/Buildings/Buildings.png}
  %\end{center}

  %\end{multicols}

\end{frame}

%-----------END BUILDINGS OVERVIEW-----------

%----------BEGIN BUILDINGS TYPES----------

\begin{frame}

  \frametitle{Buildings - Types}

  \begin{itemize}
  \item \textbf{Stadium} - High cost to fortify, but offers a high population capacity
  \item \textbf{Factory} - Good for starting resources, and has a high base integrity.
  \item \textbf{Apartment} - Low upgrade cost, decent population capacity, and also good
    for starting resources
  \item \textbf{House} - Easy to upgrade, but doesn't offer much else. Good for the early
    game.
  \item \textbf{Farm} - Difficult to upgrade, but offers more resources, beneficial for end
    game.
\end{itemize}

\end{frame}

%-----------END BUILDINGS TYPES-----------

\section{Score}

\subsection{Building Integrity}

%----------BEGIN INTEGRITY----------

\begin{frame}

  \frametitle{Building Integrity}

  \begin{center}
    {\large How many survive the battle?}
  \end{center}

  \begin{outline}
    \1 Each building has a structural integrity rating.
    \1 This rating can be improved by expending resources.
    \1 The likelihood of a building's survival is directly related to its
    structural integrity rating.
    \1 If a building is destroyed, everything inside it is lost - both people
    and resources.
  \end{outline}

  \begin{center}
    \includegraphics[scale=2.0]{../Images/Apartments.png}
  \end{center}

\end{frame}

%-----------END INTEGRITY-----------

%----------BEGIN SCORE----------

\begin{frame}

  \frametitle{Score}

  \begin{center}
    {\large How many survive the war?}
  \end{center}

  \begin{outline}
    \1 The remaining resources are consumed by the remaining population.
    \1 One resource is consumed per citizen per month.
    \1 The number of initial survivors, number of months survived, and the
    remaining survivors are all part of the player's score.
  \end{outline}

  \begin{center}
    \includegraphics[width=0.5\textwidth]{../Images/score.png}
  \end{center}

\end{frame}

%-----------END SCORE-----------


\end{document}
